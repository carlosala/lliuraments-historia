\documentclass[a4paper, 11pt]{article}
\usepackage[catalan]{babel}
\usepackage[left=3cm,right=3cm,top=2cm,bottom=2cm]{geometry}
\usepackage{parskip}
\usepackage{biblatex, csquotes}
\usepackage{amsmath, amssymb}
\usepackage{float, graphicx}
\usepackage{bookmark}

\addbibresource{l05.bib}

\hypersetup{
  colorlinks=true,
  citecolor=magenta,
  linkcolor=blue,
  urlcolor=cyan,
  pdftitle={Interpretació geomètrica de YBC6967},
  pdfpagemode=FullScreen,
}

\title{L'axiomàtica de Hilbert}
\author{
  Carlo Sala Gancho\\
  Història de les Matemàtiques\\
  Grau de Matemàtiques\\
  Universitat Autònoma de Barcelona}

\date{Maig 2023}

\begin{document}

\maketitle

\subsection*{Què és un axioma?}

Abans d'entendre el per què de tot plegat, cal entendre què és un axioma. Un axioma~\cite{bib:wiki:axioma} és una
assumpció que es pren com a certa i innegable alhora de definir les bases d'una ciència. Sovint és una propietat bàsica
que es reconeix com a certes però que manca d'un raonament lògic que ens porti a ella. El gran exemple d'axiomes i
demostracions a partir d'aquests els trobem amb Euclides a \textit{Els elements}.

\subsection*{L'axiomatització de Hilbert}

En aquest text, Hilbert declara la seva intenció, gairebé obsessiva, de generalitzar i oferir lleis universals per tot
allò que l'envolta. Parla des d'estats, fins a matemàtiques, passant per altres ciències, etc. Hilbert se n'adona que
si parem atenció en una teoria, sigui quina sigui la seva procedència, trobarem un conjunt de proposicions particulars
que formen el substrat necessari per a la construcció de tota la resta de conceptes d'aquesta teoria en particular.
Aquesta idea de trobar aquest conjunt d'axiomes és el punt principal del seu text, mostrant multitud d'exemples de com
diverses branques de les matemàtiques troben aquestes bases.

El propi Hilbert va definir 20 axiomes~\cite{bib:wiki:hilbert} que donen la base de la geometria euclidiana des del
punt de vista de la matemàtica moderna. Hilbert va crear tota una escola i un corrent de matemàtics a l'inici del segle
XX, però a nivell purament d'aportació a la ciència, segurament aquest sigui el seu treball més reconegut fins al dia
d'avui. Sens dubte, Hilbert va dedicar part de la seva recerca a ententre com crear un paquet d'axiomes que, en efecte,
fossin necessaris i suficients per al correcte desenvolupament de la geometria, i complissin amb els ja esmentats
prèviament per Euclides.

\subsection*{Té sentit axiomatitzar-ho tot?}

Hilbert defensa que té sentit axiomatitzar-ho tot. Des del meu punt de vista, crec que té molt sentit dins d'una
disciplina tan centrada en la lògica com ho és la matemàtica. Segurament, altres ciències menys lògiques i, d'alguna
manera, binàries no té tant de sentit. D'alguna manera, Hilbert peca d'egocentrisme matemàtic en aquest aspecte deixant
de banda altres disciplines.

El fet de tenir axiomes fa que treballar matemàticament sobre qualsevol àmbit sigui molt còmode, perquè saps exactament
què i què no estàs \textit{autoritzat} a fer. Tanmateix, també cal dir que per poder establir unes bases primer cal
haver estudiat i aprofondit molt en una matèria. Van fer falta molts segles perquè Hilbert pogués definir els encara
vigents axiomes de la geometria, és per això que no pot ser l'objectiu inicial d'una teoria.

Hilbert reivindica que les matemàtiques estan destinades a ocupar una posició d'importància dins de la ciència, degut a
aquesta relació tan especial amb l'axiomàtica. Des d'aquest punt de vista, no crec que sigui absolutament essencial
axiomatitzar-ho tot, com a mínim fora de les matemàtiques. La posició de dominància (o no) d'una ciència sobre una
altra no hauria d'estar únicament fonamentada sobre el grau d'axiomatització.

Cal entendre el moment històric en el que Hilbert redacta aquest text. La primera part del segle XX és un període
d'especialització dins de la matemàtica i això fa que les diverses branques posin de manifest les seves bases per poder
construir tots els professionals del sector de forma comuna i ordenada.

\printbibliography{}
\end{document}
