\documentclass[a4paper, 11pt]{article}
\usepackage[catalan]{babel}
\usepackage[left=3cm,right=3cm,top=2cm,bottom=2cm]{geometry}
\usepackage{amsmath, amssymb}
\usepackage{float, graphicx}
\usepackage{bookmark}

\hypersetup{
  colorlinks=true,
  citecolor=magenta,
  linkcolor=blue,
  urlcolor=cyan,
  pdftitle={Interpretació geomètrica de YBC6967},
  pdfpagemode=FullScreen,
}

\title{El raonament matemàtic d'Euclides}
\author{
  Carlo Sala Gancho\\
  Història de les Matemàtiques\\
  Grau de Matemàtiques\\
  Universitat Autònoma de Barcelona}

\date{Març 2023}

\begin{document}
\maketitle

\subsection*{Enunciat d'Euclides II-11}
Donarem l'enunciat de la traducció en català de l'historiador Josep Pla.
\paragraph*{EII-11} \textit{Volem tallar un segment [en dues parts] de manera que el rectangle determinat pel segment
  sencer i un dels segments sigui equivalent al quadrat de l'altre segment.}

\subsection*{Interpretació de l'enunciat}
Tenim en compte que ens trobem al llibre segon d'Euclides, per tant no ens sorprèn trobar una proposició de caire
geomètric. Ara bé, aquesta proposició té una particularitat que la fa sortir de la tònica habitual de l'autor i veurem
més endavant.\\
Seguint amb l'enunciat, cal notar que, reordenant una mica les paraules, Euclides busca un segment dividit en dues
parts (diferents, si són iguals l'enunciat és evidentment fals) en les quals el segment total sigui en relació al
segment gran com el segment gran és al petit\footnotemark. Aquesta relació ens és familiar. En efecte, es tracta de la
relació que actualment anomenem \textit{proporció àuria}, i que els he\lgem{}ènics de l'època euclidiana anomenaven
\textit{mitja i extrema raó}.

\footnotetext{Cometent un anacronisme greu, es deixa una nota pel lector del present per compendre millor d'on surt la
  relació. Sigui $x$ el segment sencer, i $y,z$ les parts, amb $x=y+z$ i $y > z$. L'enunciat busca els $y,z$ que fan que
  $y^2 = x \cdot z \Longleftrightarrow \frac{x}{y} = \frac{y}{z}$. Queda per tant vist que tant l'enunciat d'Euclides com
  aquest alternatiu són equivalents.}

\subsection*{El raonament matemàtic d'Euclides}
La manera com Euclides construeix aquesta demostració ens fa dona una idea bastant representativa dels Elements. Cal
diferenciar-ne dues fases. Primerament, comença amb la fase de construcció. En aquesta part no provarà res que el faci
arribar al resultat final. Simplement, partint de les hipòtesis inicials construeix alguna cosa nova sense que, a
priori, hagi de tenir cap relació (en aquest cas, partint del segment donat i només utilitzant regle i compàs,
construeix una figura geomètrica). Tot això, ho fa amb moltíssima cura, sempre veient que pot fer cada una de les
passes que fa. Veiem que, en aquest exemple, utilitza diverses proposicions del llibre primer tals com $E_I46$, $E_I10$
i el primer postulat, entre d'altres.\\
La segona fase és la de demostració. En aquest cas sí que provarà certes propietats del que ha construit en la fase
anterior per acostar-se i, eventualment, demostrar allò que s'havia proposat. La cura amb la qual fa cada passa es
manté com una constant en tots els llibres i totes les fases de demostració.

\subsection*{La particularitat del II-11}
En els Elements, veiem com cada una de les proposicions ens van acostant cada vegada més cap a la següent, i són
impossibles de compendre sense les anteriors. La proposició II-11 n'és una excepció. Veiem com, en la seva demostració,
fa servir gairebé només proposicions, definicions i nocions comunes del llibre primer\footnote{Cal remarcar que, en un
  cert punt, usa $E_{II}6$}. A més, aquesta proposició no s'usa per demostrar-ne d'altres, cosa que sembla trencar una
mica l'organització de l'obra. Tanmateix, trobem que aquesta proposició és l'excepció que confirma la regla, i tan bon
punt la culmina continua amb la dinàmica habitual.

\subsection*{Conclusió}
Euclides construeix una manera de pensar i fer matemàtiques realment digna de menció, i que el porta a escriure l'obra
més editada de tots els temps, només superat per la Bíblia. Aquesta manera de fer ciència és la que ens ha arribat:
construir el coneixement a sobre del ja existent és el que porta a la humanitat a evolucionar i revolucionar la ciència
de forma tan ràpida i eficient. Sens dubte cal ser conscients que el mètode que Euclides ens va proposar ens ha fet no
només aprendre matemàtiques, sinó aprendre a crear-les.

\end{document}
