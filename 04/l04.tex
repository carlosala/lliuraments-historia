\documentclass[a4paper, 11pt]{article}
\usepackage[catalan]{babel}
\usepackage[left=3cm,right=3cm,top=2cm,bottom=2cm]{geometry}
\usepackage{parskip}
% \usepackage{biblatex, csquotes}
\usepackage{amsmath, amssymb}
\usepackage{float, graphicx}
\usepackage{bookmark}

\hypersetup{
  colorlinks=true,
  citecolor=magenta,
  linkcolor=blue,
  urlcolor=cyan,
  pdftitle={Interpretació geomètrica de YBC6967},
  pdfpagemode=FullScreen,
}

\title{La professionalització de la matemàtica segons Felix Klein}
\author{
  Carlo Sala Gancho\\
  Història de les Matemàtiques\\
  Grau de Matemàtiques\\
  Universitat Autònoma de Barcelona}

\date{Maig 2023}

\begin{document}

\maketitle

\subsection*{La matemàtica, una professió?}
  Fins els volts de l'any 1800, la paraula matemàtic no era compresa com una professió, com ho podia ser arquitecte,
  banquer o comptable. I, en efecte, totes aquestes tres professions usaven les matemàtiques per desenvolupar els seus
  afers, però sempre hi havia un objectiu superior que mai era la matemàtica en última instància.\\
  Cal remarcar el context històric en el qual ens trobem. El creixement dels estats-nació, amb una gran despesa en I+D
  per ser a l'avantguarda de la tècnica i l'enginyeria i no quedar-se endarrere, propicien que les escoles tècniques, o
  \textit{hochschule} en alemany, creixin amb grans subvencions tant estatals com privades. La matemàtica és un dels
  diversos camps d'estudi que es professionalitza en aquesta particular situació.

\subsection*{La locució llatina \textit{pia fraus}}

  Traduida al català, vol dir frau o mentida piadosa. És aquell tipus d'enganys, omissions de veritat o mentides
  pròpiament que s'executen per evitar patiment, incomprensió o bé simplement simplificar la vida del que la rep. S'ha
  fet servir molt en camps com la medicina, on la família juntament amb l'equip mèdic amaguen un diagnòstic a un pacient
  per evitar que el pacient sofreixi patiments innecessaris.\\
  Ara bé, Felix Klein hi fa referència d'una manera molt particular.

\subsection*{La \textit{pia fraus} de Klein}

  Fins al segle XIX, el fet que la matemàtica no fos professional permetia que algú sense coneixements (però amb molts de
  temps, ganes i recursos) pogués posar-se al dia dels avenços matemàtics sense gaire dificultat. D'ençà que els
  matemàtics esdevenen professionals, la matemàtica avança a un ritme vertiginós i fa molt difícil d'entendre per un
  lector format en matemàtiques, però no professional, els seus últims avenços.\\
  En relació a aquesta dificultat, Klein proposa restringir al mínim el formalisme en els textos matemàtics i
  ressaltar-ne el progrés històric de la mateixa. També escriu que sovint l'interès dels humans es queda en la superfície
  del que observem, i l'objectiu és de captar-ne les idees principals per aconseguir trobar un petit resum de la bellesa
  i particularitats de cada cosa. A més, afirma que amb la creixent quantitat de matemàtics professionals, ni tan sols
  una ment privilegiada pot estar completament actualitzada i comprenent ben bé el que succeeix en cada moment de les
  investigacions. Aquest conjunt de raons porten a Klein a acceptar que cal simplificar el discurs (com a mínim de forma
  frontal i com a punt d'entrada) amb l'objectiu de difondre encara més el missatge matemàtic. Aquesta és la \textit{pia
    fraus} de Klein.

\subsection*{Té raó Klein?}
  El caràcter pragmàtic de Klein és molt notable en aquestes pàgines d'introducció a les seves \textit{Lliçons sobre el
    desenvolupament de les matemàtiques al segle XIX}. Personalment, penso que la divulgació matemàtica és el punt
  d'entrada per a molts inexperts en aquesta ciència a interessar-se i, fins i tot, participar del seu desenvolupament.
  La nostra ciència pot ser tot sovint esfereïdora d'entrada, però comprenent-ne la seva bellesa i ordre fa que pugui ser
  apreciada per una gran varietat de persones. Des del meu punt de vista, Klein encerta de ple. Tanmateix, crec que no
  cal perdre els estreps i passar-se de frenada: cal seguir desenvolupant la matemàtica d'una manera estricta, ordenada,
  plena i fonamentada alhora que no es perd de vista la importància la divulgació dels coneixements.

\end{document}
