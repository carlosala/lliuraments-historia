\documentclass[a4paper, 11pt]{article}
\usepackage[catalan]{babel}
\usepackage[left=3cm,right=3cm,top=2cm,bottom=2cm]{geometry}
\usepackage{biblatex, csquotes}
\usepackage{parskip}
\usepackage{amsmath, amssymb}
\usepackage{float, graphicx}
\usepackage{bookmark}

\addbibresource{l06.bib}

\hypersetup{
  colorlinks=true,
  citecolor=magenta,
  linkcolor=blue,
  urlcolor=cyan,
  pdftitle={Interpretació geomètrica de YBC6967},
  pdfpagemode=FullScreen,
}

\title{La matemàtica soviètica}
\author{
  Carlo Sala Gancho\\
  Història de les Matemàtiques\\
  Grau de Matemàtiques\\
  Universitat Autònoma de Barcelona } \date{Maig 2023}

\begin{document}
\frenchspacing

\maketitle

\subsection*{L'era daurada de les matemàtiques soviètiques}

Abans de la Revolució de 1917, l'imperi dels tsars no va promoure de forma generalitzada l'expansió de les ciències
durant el segle XIX, com sí que ho van fer altres estats occidentals. Una Rússia encara sotmesa a les lleis feudals que
s'havien abandonat feia dècades (o segles en algunes localitzacions) dificultaven el creixement de la indústria i de la
ciència. Després de la Gran Guerra, es desenvolupa amb força fins arribar a la Segona Guerra Mundial sent una de les
tres potències del moment i, un cop acabada, es queda tota sola amb els EUA dividint-se el control del món. Comença la
guerra freda.

No ens sorprèn que en aquestes circumstàncies les matemàtiques soviètiques visquin els seus millors anys, de la mateixa
manera que va passar amb moltes altres ciències, amb l'objectiu de superar els EUA en la cursa espacial i altres
\textit{picabaralles} que tenien per fer veure l'un a l'altre qui dominava el món. Poc madur, probablement, però és un
tema tangencial que deixarem de banda. L'historiador rus (tot i que amb residència als EUA des de fa dècades, a tenir
en compte per la lectura) Slava Gerovitch~\cite{bib:wiki:slava} anomena \textit{l'era daurada} de les matemàtiques
soviètiques, dels anys 50 fins a l'inici dels 80 del segle XX. Ara bé, no és or tot allò que llueix.

\subsection*{El règim soviètic, i els seus problemes intrínsecs}

El règim soviètic es va caracteritzar l'autocràcia que provaven d'implementar va fer que sovint els científics es
veiessin afectats de retruc. Gerovitch descriu com els matemàtics soviètics patien moltes restriccions tant de
mobilitat, administratives o fins i tot de contacte i co\lgem{}aboració amb altres investigadors d'altres parts del
món. Fins un cert punt, pot semblar que aquests no són problemes greus, però també l'accés a articles i revistes
occidentals estava molt limitat i només els matemàtics més experimentats, i amb certa afinitat al règim, podien
accedir-hi. A més, Gerovitch afirma el sistema educatiu soviètic era molt rígid i exigent, sovint sense deixar a
l'estudiant triar quina branca li agradava més, o era més afí amb ell. Sovint aquests plans d'estudis quedaven
desactualitzats i era molt difícil canviar-los en un règim que pressionava tant.

La pròpia idiosincràsia del sistema va fer que sovint les matemàtiques es desenvolupessin en para\lgem{}el a les
universitats. Molts potencials matemàtics en quedaven fora per la rigidesa de l'administració i va convertir-se en una
cosa comuna el fet de practicar matemàtiques en espais relativament privats, com el seminari que va potenciar
Gelfand~\cite{bib:wiki:gelfand} a Moscou. Aquesta manera de fer matemàtiques va sorprendre a molts professionals
occidentals, que veien com grans matemàtics soviètics ho eren durant el seu temps lliure, sovint combinant-ho amb
altres feines i, d'alguna manera, sent les matemàtiques el seu \textit{hobby}.

\subsection*{Les matemàtiques soviètiques, recordant el passat?}

Aquesta manera de fer anar les matemàtiques recorden a les matemàtiques de l'Edat Moderna. El gran exemple d'això és
Leibniz, que tenia la seva feina com a diplomàtic, tota una família, i encara així trobava temps per ser un dels
matemàtics més grans de tots els temps, i fer avenços esfereïdors en les matemàtiques. D'alguna manera, una part dels
matemàtics soviètics també tenien aquesta relació informal amb la ciència. Tanmateix, durant l'Edat Moderna aquesta era
l'única manera de fer matemàtiques ja que encara no s'havia professionalitzat la disciplina, i a l'època soviètica els
grans líders matemàtics del moment que participaven d'aquests seminaris i grups de recerca para\lgem{}els a la
universitat sí que hi estaven vinculats. D'alguna manera, van ser obra del sistema.

Des del nostre punt de vista occidental, les matemàtiques soviètiques són estranyes en el sentit de com estaven
estructurades. Cal comprendre el sistema soviètic en conjunt per poder entendre el perquè de com van funcionar, sempre
molt marcades per la rigidesa i l'estatalització de les institucions educatives.

\printbibliography{}

\end{document}
